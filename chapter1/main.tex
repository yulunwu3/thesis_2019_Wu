\chapter{Introduction}
Large power VHF/UHF radar systems used with very large gain antennas in ionospheric research are known as incoherent scatter radar (ISR). The highest sensitivity ISR in operation in the world today is located at the Arecibo Observatory in Puerto Rico, and this thesis presents the coded long pulse (CLP) spectral estimation and analysis using measurement data from the Arecibo ISR from ionospheric altitudes above about 75 km.

The mechanism underlying incoherent scattering in ISR operations is the ``dipole radiation'' of each free electron in the ionosphere made to oscillate by the transmitted radar pulse -- this is known as the Thomson scattering process. The density of Thomson scttering free electrons in the ionospheres fluctuations as a superposition of electron density waves propagating in all directions across a broad spectrum of wavelengths with propagation velocities governed by the Langmuir and ion-acoustic wave dispersion relations. An ISR will only detect the superposition dipole radiation (Thomson scattering) signals of the electrons ``belonging to'' density waves whose wavefronts are perpendicular to the radar beam since scattering from electrons of waves propagating in other directions will be self-cancelling due to the destructive interference. Furthermore, only the scattering of the electrons of waves with a wavelength equal to one half of the wavelength of the transmitted radar pulse will not self-cancel -- this wave component solely responsible for the backscattered radar signal is called the ``Bragg wave''. The operation frequency of the Arecibo ISR is 430 MHz, which corresponds to about 70 cm wavelength, and 35 cm Bragg wavelength, meaning that Arecibo ISR will only detect signals returned from 35 cm wavelength electron density waves which are propagating parallel or anti-parallel to the direction of radar beam. The Arecibo ISR signal spectrum will then exhibit a pair of peaks up-shifted from 430 MHz, each one caused by 35 cm Bragg waves propagating toward the radar at the ion-acoustic velocity $C_s$ as well as the plasma-wave phase shift speed $\omega_p/k_B$, respectively, where $\omega_p$ is the plasma frequency and $k_B$ is the Bragg wavenumber, in addition to a pair of down-shifted peaks caused by the same waves propagating in the opposite direction. The slower phase velocity peaks in the ISR spectrum relate to ion-acoustic waves in the ionospheric plasma while the fast phase velocity peaks represent electron plasma (Langmuir) waves. Landau damping and collisional damping of the scattering density waves will contribute to the broading of eac of these spectral peaks. The broadened low-frequency ion-acoustic peaks will tend to merge together to form the ``double humped'' ion-line feature of the ISR spectrum. The shape of this ``double-humped'' ion-line componentes has been derived by \textit{Kudeki and Milla} [2011], and is essentially a superposition of electron and ion velocity distribution functions scaled by $k_B$ and frequency-dependent weighting coefficients describing the collective interactions of ionospheric charged particles (electrons and ions) via polarization electric fields that they cause.

(Next we describe coded long pulse here, can borrow something from senior thesis)

There are six additional chapters in this thesis:
\begin{itemize}
    \item Chapter 2 introduces basic concepts about the ionosphere and presents the derivation of the model equation of the spectrum of the electron density fluctuations causing the radar backscater -- a convolutionally distorted version of the spectrum also models the backscattered radar signal spectrum that can be computed from the sampled radar data.
    \item Chapter 3 describes the configuration of the Arecibo ISR system and the technologies utilized in the experiment using CLP data collection mode.
    \item Chapter 4 presents the computation of ISR ion-line power spectrum from CLP raw voltage data using fast fourier transform (FFT) and point spread function (PSF) that is used in inversion of convolutionally distorted power spectra.
    \item Chapter 5 presents the details of the theory of ion-line spectrum including the electron and ion Gordeyev integral that involves the effects of Coulumb collision, ion-neutral collision, geomagnetic fields, and magnetic aspect angle, followed by the computation of such model using chirp-z algorithm.
    \item Chapter 6 presents the inversion of electron and ion temperatures from CLP ion-line power spectra using independent fitting and profile fitting techniques. The inversion of ion velocity from auto-correlation function (ACF) of power spectra and the inversion method of electron density and radar system calibration from ion-line power spectra are discussed.
    \item Chapter 7 presents the conclusions of this study and the future directions of improving current inversion algorithm.
\end{itemize}