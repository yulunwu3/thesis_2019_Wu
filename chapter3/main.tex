\chapter{Radar configuration at arecibo observatory and computation of coded long pulse ion-line spectrogram}
The Arecibo Observatory is located near the northern coastal town of Arecibo on the island of Puerto Rico in a region populated by natural sinkholes in its terrain. One such sinkhole about 15 km inland from Arecibo houses the largest single dish spherical reflector antenna in the world used for space research. The Arecibo reflector antenna is part of the Arecibo ISR system that was designed and built by William E. Gordon of Cornell University in the mid-1960s and maintained by Cornell University until 2011. Since 2011 the Arecibo Observatory and its ISR system have been operated under coorperative agreements with Ntional Science Foundation. The Arecibo facility supports three major areas of research: radio astronomy, atmospheric science, and radar astronomy. The observatory has radar transmitters with effective isotropic radiated power of 1 MW at 2380 MHz (S-band system) and 2.5 MW at 430 MHz (the ionospheric ISR system). This thesis is focusd on the use of Arecibo ISR system for ionospheric measurements and research, specifically in coded-long-pulse data mode of the system.

\section{Arecibo ISR - System Description} The Arecibo ISR is a 430 MHz backscatter radar system using a 305 m diameter spherical dish antenna as shown in Figure ??. The radar transmitter generates pulses with peak power of 2.5 MW at the 430 MHz oprerating frequency [Isham et al., 2000]. Given its very large antenna aperture and transmitted power operated within the UHF band, Arecibo ISR achieves an overall sensitivity about 100 times larger than that of other ISR systems currently in existance. The 430 MHz line feed makes an efficient use of the main dish when pointed vertically, as its radiation pattern fills the available aperture. The new Gregorian feed that was added to the system during the 2000 upgrades enables dual beam operations for more efficient determinations of ionospheric plasma drifts [same citation here 2000].

\section{Arecibo ISR Coded Long Pulse Data Mode}
Isham et al. [2000] describes six data acquisition modes for Arecibo ISR operations. One of them, coded long pulse (CLP), is of importance in F-region and topside ionospheric studies. The use of CLP data mode for marrowband ion-line measurements with the main Arecibo ISR receiver will be described below.

In soft target radar measurements, if the correlation times of the scattering density waves are short compared to the inter-pulse period (IPP), then ``pulse-to-pulse'' correlation methods cannot be used and it becomes necessary to utilize ``within pulse'' correlation methods with multiple samples taken from the supreposed echos of transmieed pulses whose lengths need to exceed the sampling interval by some substantial margin. Such ``within pulse'' operations are generally referred to as ``long pulse'' techniques. The disadvantages of using long pulses is accepting relatively poor radar range resolution, unless the target SNR is so strong that short baud length coding can be applied to the transmitted long pulse to produce a range resolution determined by the baud length rather than the pulse length.

The coded long pulse (CLP) mode utilized at Arecibo implements this idea and works well to probe the lower F-region altitudes of the ionosphere where the electron density is realtively large and the corresponding scatter SNR is sufficient. The mode was developed by Sulzer [1986] for high-resolution ion-line measurements.