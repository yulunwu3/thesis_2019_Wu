\chapter{The ionosphere and the theory of ionospheric incoherent scattering}
\section{Earth's Ionosphere}
Earth's upper atmosphere is a partially ionized and electrically conducting plasma above about 50 km altitude. This deep layer of the atmosphere is called the ``ionosphere'' and blends into Earth's ``magnetosphere'' above an altitude of about 1000 km as a magnetized plasma and into the ``solar wind'' beyond.

Two types of classification are used to describe the properties of the ionospere:
\begin{itemize}
    \item Temperature profile: As shown in Figure \ref{fig:chapter2/figure/profile1} on the left, atmospheric temperature will decrease with height at an approximately constant rate below an altitude of about 10 km in a region known as the troposphere. About the troposphere the temperature will increase with height throughout a region known as the stratosphere. The region of decreasing temperature about the stratosphere is known as the mesosphere, above which lies a region of increasing temperature known as the thermosphere. Stratospheric temperature increase is due to the absorption of the ultraviolet portion of the solar radiation by ozone. Mesospheric temperature decrease with height above 50 km is caused by radiative cooling whereas thermospheric temperature increase is caused by dattime absorption of solar photons in UV and EUV frequency bands.
    \item Plasma density profile: This characteristic is defined as the number of free electrons per unit volume. As shown in Figure \ref{fig:chapter2/figure/profile1} on the right, ionospheric plasma density reaches its peak value at a few hundred kilometers altitude and exhibits substantial variation depending on daytime (solid curve) and nighttime (dashed curve) conditions. The daytime profile represents an equilibrium between the photo-ionozation rate and recombination rate of plasma production and decay. In daytime, the solar spectrum is incident on a neutral atmosphere whose electron density increases exponentially with decreasing altitude. Since the photons are absorbed in the process of photoionization, the incoming beam itself decreases in intensity as it penetrates the atmosphere. The combination of decreasing solar flux increasing neutral density, and diffusion provides a simple explanation for the basic large scale vertical layer of ionization [Kelley, 2009]. From 60 km to 90 km altitude is the D region, from 90 km to 150 km is the E region, and beyond 150 km lies the F region. Peak plasma density occurs in the F region. The ion composition differs among regions; at lower altitudes, namely in the D region, it is less affected by the solar radiation. Therefore, there is a large number of neutral particles in the lower altitude. In the F region, on the other hand, due to the chemical reaction between oxygen gas, nitrogen gas and solar UV radiation, $O^+$ and $N^+$ ions will dominate the ion composition [Kelley, 2009].
\end{itemize}

Ionospheric plasma density drops dramatically at night at D region heights while staying nearly the same higher up in the ionosphere in the F-region; this is due to the dramatic difference of the recombination rates of the F- and D-region plasmas as explained in Kelley [2009], where automic and molecular ions dominate, respectively. In the lower altitude D-region the recombination process may beam
\begin{align*}
    NO^+ + e^- \rightarrow N+O
\end{align*}
and 
\begin{align*}
    O_2^+ + e^- \rightarrow O+O
\end{align*}
In F-region, the dominant recombination process goes as
\begin{align*}
    O^++e^-\rightarrow O+\mbox{photon}
\end{align*}
Without the solar radiation at night, the recombination at lower altitude will occur more frequently than at higher altitude. As a result, the plasma density in D-region will almost disappear.


\section{Incoherent Scatter Theories}
According to the incoherent scatter spectral theory [Kudeki and Milla, 2011], ionospheric incoherent scatter is caused by Thomson scattering of radar pulses at radio frequencies from collections of ionospheric free electrons. In this section, we will first describe the Thomson scattering process by single electrons, then study the Thomson scttering effect of multiple electrons.
\subsection{Thomson scattering by a free electron}